%%% Lecture 2

\section{Some Consequences of Hurewicz Theorem}

\lecture[Getting rid of the basepoint.]{2021-10-13}

Before resuming with the proof of the relative Hurewicz theorem we prove an application of it, a version of Whitehead's theorem which uses homology groups in place of homotopy groups.

\begin{theorem}\label{theorem:homology-whitehead}
Let $f:X\to Y$ be a map between simply connected CW-complexes such that $f_*:H_i(X;\Z)\to H_i(Y;\Z)$ is an isomorphism for all $i\geq 0$. Then $f$ is an homotopy equivalence.
\end{theorem}

\begin{proof}
By cellular approximation we can assume $f$ cellular. Let $Z(f)=X\times[0,1]\cup_{X\times1, f}Y$ be the mapping cylinder of $f$. We recall that $Z(f
)$\leftnote{That $Z(f)$ has a CW-structure is quite obvious \tit{but} modulo point-set subtleties!} inherits a CW-structure such that $X\cong X\times0$ and $Y$ are subcomplexes. The projection $Z(f)\to Y$ is a homotopy equivalence\rightnote{Proving that $Y$ is a deformation retract of $Z(f)$ might be slightly annoying because of quotient spaces' quirkiness. The idea is very straightforward, as one can construct an homotopy by sliding points of $X\times[0,1]$, but some care is needed to show that this is continuous (see \cite[A.17]{hatcher}).}, hence by replacing $Y$ by $Z(f)$ we can assume without loss of generality that $f:X\to Y$ is the inclusion of a subcomplex. Since $X$ and $Y$ are simply-connected, the relative Hurewicz theorem applies for all $n\geq2$, but all relative homology groups vanish because $f_*$ is an isomorphism (by the long exact sequence), hence all relative homotopy groups vanish and by Whitehead's theorem we can conclude that $f$ is an homotopy equivalence.
\end{proof}

An elaboration of the previous results leads to the following proposition.

\begin{proposition}
Let $f:X\to Y$ be a map of path-connected CW-complexes. The following are equivalent:
\begin{itemize}
    \item[(i)] $f$ is a homotopy equivalence,
    \item[(ii)] $f$ induces an isomorphism on fundamental groups and the induced map $\til{f}:\til{X}\to\til{Y}$ on universal covers induces an isomorphism on all integral homology groups.
\end{itemize}
\end{proposition}

\begin{proof}
$(i)\implies(ii)$ Since $f$ is an homotopy equivalence, $f_*$ is an isomorphism on all homotopy groups. Then $\til{f}$ induces an isomorphism on all homotopy groups, hence it is an homotopy equivalence, thus it induces an isomorphism on all homology groups.

$(ii)\implies(i)$ Since $\til{f}$ induces an isomorphism on integral homology groups it is a homotopy equivalence by the version of Whitehead's theorem we just proved, hence it induces an isomorphism on all homotopy groups. This in turn means that $f$ induces an isomorphism on all homotopy groups, i.e. it is a homotopy equivalence.
\end{proof}

\section{Self-Maps of Spheres and Getting Rid of the Basepoint}

We now return to the proof of the relative Hurewicz theorem.

The first ingredient in the (very long) proof of Hurewicz theorem is a lemma about self-maps of spheres that will be used in the proof of the Homotopy Addition Theorem (\ref{theorem:HAT}).

Recall: the \textbf{degree} of a map $f:(D^n,\de D^n)\to(D^n,\de D^n)$ is the integer $\deg(f)$ such that $f_*(x)=\deg(f)x$ for all $x\in H_n(D^n,\de D^n;\Z)$.

\begin{lemma}\label{lemma:degree-of-sphere-self-maps}
Let $n\geq1$. For $n>1$ assume known that $\pi_\ni(\sni,z)$ is free abelian of rank $1$. Let $f$ be a continuous self map of $\sph$ of degree $\pm 1$. Then $f$ is pair-homotopic to the identity if $\deg(f)=1$ and to any reflection if $\deg(f)=-1$.
\end{lemma}

\begin{proof}
We first see the case when $\deg(f)=1$.

$(n=1)$ Since $\de D^1=\cb{\pm1}$ and $\deg(f)=1$ we have $f|_{\de D^1}=\id_{\de D^1}$. Then the linear homotopy $H(x,t)=tf(x)+(1-t)x$ is a relative homotopy between $f$ and the identity $\id_{D^1}$.

$(n\geq2)$ Consider the commutative square:
\begin{center}
    \begin{tikzcd}
    H_n(\dn,\de\dn;\Z) \arrow[d,"f_*=\id"] \arrow[r, "\de", "\cong" below] & H_\ni(\sni;\Z) \arrow[d, "(f|_{S^\ni})_*=\id"] \\
    H_n(\dn,\de\dn;\Z) \arrow[r, "\de", "\cong" below] & H_\ni(S^\ni;\Z)
    \end{tikzcd}
\end{center}
Since $\pi_\ni(S^\ni,z)$ is free of rank $1$, the Hurewicz map $h:\pi_\ni(S^\ni,z)\to H_\ni(S^\ni;\Z)$ is an isomorphism. Then $(f|_{\de\dn})_*:\pi_\ni(\sni,z)\to\pi_\ni(\sni,z)$ is the identity, therefore $f|_{\de\dn}$ is homotopic to the identity of $\sni$. Now let $H:\sni\times [0,1]\to\sni$ be a homotopy, this gives a map
\[D^n\times0\cup\sni\times[0,1]\cup \dn\times1\xto{f\cup H\cup\id}\dn.\]
Since $\dn\times[0,1]$ can be obtained from $D^n\times0\cup\sni\times[0,1]\cup \dn\times1$ by attaching an $(n+1)$-cell and $\dn$ is contractible\normalmarginpar\marginnote{\footnotesize Why do we need $\dn$ contractible? This is essentially the fact that maps $S^n\to X$ are homotopic to the constant map if and only if they extend to a map $D^{n+1}\to X$.}, there is a continuous extension $\bar{H}:\dn\times[0,1]\to\dn$. This is the desired pair homotopy from $f$ to $\id_{\dn}$.

If $\deg(f)=-1$, we let $r:\dn\to\dn$ be the reflection in the first coordinate. Then $\deg(r\circ f)=1$, hence $r\circ f$ is pair homotopic to $\id_{\dn}$ and so $f=r\circ r\circ f$ is pair homotopic to $r$.
\end{proof}

The next step is to reduce the statement of the Hurewicz theorem to a statement that does not involve the basepoint.

Let $(X,A)$ be a based space. Define the group $\pi_n(X,A)^\#$ as the quotient of the free abelian group generated by pair homotopic maps $(I^n,\de I^n)\to (X,A)$ by the relation $[f]+[f']=[f+f']$ when the right hand side is defined.

The "forgetful" map $\pinr\to\pi_n(X,A)^\#$ is a group homomorphism and it factors through a homomorphism $\pinr^\dagger\to\pi_n(X,A)^\#$ (because $\omega * f$ and $f$ are always pair homotopic).

\begin{proposition}
Let $(X,A)$ be a pair of path-connected spaces. Let $n\geq2$ or $n=1$ and $A$ a point. Then the "forgetful" homomorphism $\pinr^\dagger\to\pi_n(X,A)^\#$ is an isomorphism.
\end{proposition}

\begin{proof}
We will define a homomorphism in the opposite direction. Let $f:(I^n,\de I^n)\to (X,A)$ be a pair map. $f$ need not send $J^\ni$ to $x_0$, but $J^\ni$ is contractible and $A$ path-connected. So $f|_{J^\ni}$ is homotopic in A to the constant map at the basepoint. Let $H:J^\ni\times[0,1]\to A$ be such a homotopy from $f|_{J^\ni}$ to the constant map $x_0$. The HEP for $(\de I^n,J^\ni)$ lets us extend $H$ to a homotopy $H':\de I^n\times[0,1]\to A$ from $f|_{\de I^n}$ to some map $H'(1,-)$ that sends $J^\ni$ to $x_0$. The HEP for $(I^n,\de I^n)$ with target space $X$ lets us extend $H'$ to $H'':I^n\times[0,1]\to X$ from $f$ to a map that sends $J^\ni$ to $x_0$. Moreover $H''$ is a pair homotopy of maps $(I^n,\de I^n)\to (X,A)$.

We now define a map \[\Psi:[(I^n,\de I^n)\to (X,A)]\to \pi_n(X,A,x_0)^\dagger\]
by sending $[f]$ to $[H''(-,1)]$. We claim that this is well defined.
 
Claim. Let $f,f':(I^n,\de I^n, J^\ni)\to(X,A,x_0)$ be triple maps that are pair homotopic as maps $(I^n,\de I^n)\to (X,A)$. Then they represent the same element in $\pi_n(X,A,x_0)^\dagger$.

\begin{claimproof}
Let $H:I^n\times[0,1]\to X$ be a pair homotopy from $f$ to $f'$. We choose a point $z\in J^\ni$ and a triple homotopy
\[K:\triple\times[0,1]\to\triple\]
from the identity to a map with $K(J^\ni\times 1)=\{z\}$. Formally what we are applying two times the HEP (for $(\de I^n,J^\ni)$ and for $(I^n,\de I^n)$, as before). Then $f$ is triple homotopic to $f\circ K(-,1)$, $f'$ is triple homotopic to $f'\circ K(-,1)$, so that $f\circ K(-,1)$ is pair homotopic to $f'\circ K(-,1)$. In particular $\til{H}=H(K(-,1),-)$ satisfies $\til{H}(J^\ni,t)=H(z,t)$ for all $t\in [0,1]$. Now for all $x\in J^\ni$, the loop at $x_0$ in $A$, $\til{H}(x,-)$, is independent of the point $x\in J^\ni$ and it always agrees with $\omega=H(z,-)$. By reparametrizing $I^{n+1}$\alvaropls\ we can view $\til{H}$ as a triple homotopy between $\omega * (f\circ K(-,1))$ and $f'\circ K(-,1)$. In the end we have
\[[f]=[f\circ K(-,1)]=[\omega * (f\circ K(-,1))]=[f'\circ K(-,1)]=[f']\] in $\pinr^\dagger$ (the second equality holds in $\pinr^\dagger$ by construction, the other ones are all homotopies we constructed).
\end{claimproof}

As a result of the claim the map $\Psi:[(I^n,\de I^n),(X,A)]\to\pinr^\dagger$ is well-defined, so it has a unique extension on the free abelian group which factors to a homomorphism $\pi_n(X,A)^\#\to\pinr^\dagger$ which is then an isomorphism by design.
\end{proof}
 
Punchline: In the situation of the relative Hurewicz theorem it suffices to show that the map $\pi_n(X,A)^\#\to H_n(X,A;\Z)$ is an isomorphism (i.e. we don't have to deal with basepoints!).
