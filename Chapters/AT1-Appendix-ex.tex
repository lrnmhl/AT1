%%% Appendix

\chapter{Appendix}

\section{Exercises}

In this section\rightnote{This is a work in progress...} we collect the weekly exercises we were given during the semester, sometimes (for the important or nice ones) with sketches of the proofs.

\subsection{AT1Sheet1}

The first exercise is about an inherently interesting result.

\unnumpar{AT1Sheet1.1}\label{exercise:AT1Sheet1.1}
Let $X$ be a finite CW-complex and $Y$ a space such that for every basepoint $y\in Y$ and every number $i$ that is less than or equal to the dimension of $X$ the set $\pi_i(Y,y)$ is finite. Show that the set $[X,Y]$ of homotopy classes of maps from $X$ to $Y$ is finite.

\begin{sketch}
\todo[inline,color=yellow]{To be added}
\end{sketch}

In the first lecture \hyperref[paragraph:hurewicz-morphism]{we introduced the pinch map}, which can be used to describe the group structure on the higher homotopy groups. In the second exercise, we describe such a map, we prove its key property (needed to show that the Hurewicz map is a morphism of groups) and we show that it is unique up to homotopy.

\unnumpar{AT1Sheet1.2}\label{exercise:AT1Sheet1.2}
For $n \ge 1$ consider the maps
\[q_1, q_2 : S^n\vee S^n\to S^n\]
where $q_i$ is the identity on the $i$-th wedge summand and sends the other wedge summand to the basepoint. A \tbf{pinch map} is a continuous map $p : S^n\to S^n\vee S^n$ such that both
composites $q_1 \circ p$ and $q_2 \circ p$ are homotopic to the identity map of $S^n$.
\begin{enumerate}
    \item[(a)] Show that there is a pinch map for every $n\ge 1$.
    \item[(b)] Show that the effect of a pinch map on singular homology is given by
    \[ p_*(x) = (i_1)_*(x) + (i_2)_*(x)\]
    for all coefficient groups $A$ and all $x\in H_n(S^n,A)$, where $i_1, i_2 : S^n \to S^n\vee S^n$ are the two wedge summand inclusions.
    \item[(c)] Suppose now that $n\ge 2$. Use the Hurewicz theorem to show that any two pinch maps are homotopic as based maps.
\end{enumerate}

\begin{sketch}
\todo[inline,color=yellow]{To be added}
\end{sketch}

The next exercise serves as a reality check: the second relative higher homotopy group is not, in general, abelian, but the quotient of it we consider in the proof of the relative Hurewicz theorem is.

\unnumpar{AT1Sheet1.3}\label{exercise:AT1Sheet1.3}
Let $A$ be a subspace of a space $X$ and let $\pi_2(X,A,x_0)^\dagger$ be the factor
group of $\pi_2(X,A,x_0)$ by the normal subgroup generated by all elements of the form \[(\omega * f) \cdot f^{-1}\]
for all $\omega\in\pi_1(A,x_0)$ and all $f\in\pi_2(X,A,x_0)$. Show that the factor group $\pi_2(X,A,x_0)^\dagger$ is abelian.

\begin{sketch}
\todo[inline,color=yellow]{To be added}
\end{sketch}

\subsection{AT1Sheet2}

An example of a non-trivial fundamental group action.

\unnumpar{AT1Sheet2.1}\label{exercise:AT1Sheet2.1}
This exercise shows, among other things, that the higher homotopy groups of a finite CW-complex need not be finitely generated (in contrast to the integral homology groups).
Let $X = S^{1} \vee S^{2}$ be the onepoint union of a circle and a $2$-sphere.
\begin{enumerate}
    \item[(a)] Find a universal cover of $X$ and calculate $\pi_{1}(X, x_{0})$ as the deck transformation group of the universal cover.
    \item[(b)] Use the Hurewicz theorem for the universal cover to calculate the group $\pi_{2}(X, x_{0})$.
    \item[(c)] Define a family of continuous maps $S^{2} \rightarrow X$ whose homotopy classes form a basis of $\pi_{2}(X, x_{0})$. Determine the action of the fundamental group $\pi_{1}(X, x_{0})$ on $\pi_{2}(X, x_{0})$ in terms of your basis.
\end{enumerate}

\begin{sketch}
\todo[inline,color=yellow]{To be added}
\end{sketch}

An application of Hurewicz theorem.

\unnumpar{AT1Sheet2.2}\label{exercise:AT1Sheet2.2}
Let $X$ be an acyclic CW-complex, i.e., all reduced integral homology groups of $X$ are trivial. Show that the suspension of $X$ is contractible.

\begin{sketch}
\todo[inline,color=yellow]{To be added}
\end{sketch}

In the following exercise we study the mapping telescope, a construction which is related to the notion of homotopy colimit.

\unnumpar{AT1Sheet2.3}\label{exercise:AT1Sheet2.3}
Let 
\begin{center}
\begin{tikzcd}
X_{0} \arrow[r, "f_{0}"] & X_{1} \arrow[r, "f_{1}"] & \cdots \arrow[r, "f_{n - 1}"] & X_{n} \arrow[r, "f_{n}"] & \cdots
\end{tikzcd}
\end{center}
be a sequence of topological spaces and continuous maps. The \textit{mapping telescope} of the sequence is the space
\begin{equation*}
    \tel_{n} X_{n} = \Bigl\{ \bigsqcup_{n \geq 0} X_{n} \times [0, 1] \Bigr\} / \sim,
\end{equation*}
where $\sim$ denotes the equivalence relation generated by
\begin{equation*}
    (x, 1) \sim (f_{n}(x), 0)
\end{equation*}
for all $n \geq 0$ and all $x \in X_{n}$. We define continuous maps
\begin{equation*}
    i_{k} : X_{k} \rightarrow \tel_{n} X_{n}
\end{equation*}
be letting $i_{k}(x)$ be the equivalence class of $(x, 0)$.
\begin{enumerate}
    \item[(a)] Show that the maps
    \begin{equation*}
        (i_{k})_{*} : H_{*}(X_{k}, A) \rightarrow H_{*}(\tel_{n}X_{n}, A)
    \end{equation*}
    induce an isomorphism
    \begin{equation*}
        \colim_{k}H_{*}(X_{k}, A) \rightarrow H_{*}(\tel_{n} X_{n}, A)
    \end{equation*}
    for every abelian coefficient group $A$.
    \item[(b)] Let $x_{0} \in X_{0}$ be a basepoint and define $x_{n} \in X_{n}$ inductively by $x_{n} = f_{n - 1}(x_{n-1})$. Use the maps
    \begin{equation*}
        (i_{k})_{*} : \pi_{m}(X_{k}, x_{k}) \rightarrow \pi_{m}(\tel_{n}X_{n}, (i_{k})(x_{k}))
    \end{equation*}
    to define an isomorphism
    \begin{equation*}
        \colim_{k} \pi_{m}(X_{k}, x_{k}) \rightarrow \pi_{m}(\tel_{n}X_{n}, (i_{0})(x_{0}))
    \end{equation*}
    Some care has to be taken here with basepoints ...
\end{enumerate}

\begin{sketch}
\todo[inline,color=yellow]{To be added}
\end{sketch}

\subsection{AT1Sheet3}

There is a fundamental adjunction between the geometric realization functor and the singular simplicial set functor that we will use throughout the course (especially in chapters \ref{chapter:hurewicz} and \ref{chapter:the-cool-chapter}).

\unnumpar{AT1Sheet3.1}\label{exercise:AT1Sheet3.1}
Let $X$ be a simplicial set, $T$ a topological space and $f : X\to \mathcal{S}(T)$ a
morphism of simplicial sets, where $\mathcal{S}(-)$ denotes the singular simplicial set.
\begin{itemize}
    \item[(a)] Consider the continuous map
    \begin{align*}
        \bigsqcup_{n\ge0} X_n\times\nabla^n &\to T\\
        X_n\times\nabla^n\ni (x,t) &\mapsto f_n(x)(t).
    \end{align*}
.    Show that this induces a continuous map $\widehat{f} : |X| \to T$ defined on the geometric realization.
    
    \item[(b)] Show that for every simplicial set $X$ and every topological space $T$ the assignment
    \begin{align*}
    \Hom_{\text{simpl. sets}}(X,\mathcal{S}(T)) &\to \Hom_{\text{top. spaces}}(|X|,T)\\
    f &\mapsto \widehat{f}
    \end{align*}
is bijective.
    
    \item[(c)] Show that the bijection of part (b) is natural in both variables.
\end{itemize}

\begin{sketch}
\todo[inline,color=yellow]{To be added}
\end{sketch}

In the second exercise we prove a very useful homeomorphism which is used in the proof of theorem \ref{theorem:simplicial-deformation-retraction} and will be generalized in \hyperref[exercise:AT1Sheet11.2]{AT1Sheet11.2}.

\unnumpar{AT1Sheet3.2}\label{exercise:AT1Sheet3.2}
Show that for every $n\ge0$ the map
\[
(|p_1|,|p_2|) : |\Delta[n] \times \Delta[1]| \to |\Delta[n]|\times|\Delta[1]|
\]
is a homeomorphism, where $p_1 : \Delta[n] \times\Delta[1]\to\Delta[n]$ and $p_2 : \Delta[n] \times\Delta[1] \to\Delta[1]$ are the
projections to the two factors.

\begin{sketch}
\todo[inline,color=yellow]{To be added}
\end{sketch}

The nerve construction is a way of constructing simplicial sets from small categories (in \hyperref[exercise:AT1Sheet4.1]{AT1Sheet4.1} we will use it to construct a simplicial set out of a group).

\unnumpar{AT1Sheet3.3}\label{exercise:AT1Sheet3.3}
Let $I$ be a small category. The nerve of $I$ is the simplicial set $NI$ given by
\[
(NI)_n = \text{ set of all composable $n$-tuples of morphisms in } I.
\]
\begin{enumerate}
    \item[(a)] Show that there is a unique way to extend this data to a simplicial set in such a way that the face and degeneracy morphisms are given by the following formulas. For $n\ge1$ and $0\ge i\ge n$, the $i$-th boundary map $d_i : (NI)_n \to (NI)_{n-1}$ is defined by
    \[d_i(f_n,...,f_1) = \begin{cases}
        (f_n,...,f_2) & i = 0,\\
        (f_n,...,f_{i+2},f_{i+1}\circ f_i,f_{i-1},...,f_1) &0 < i < n,\\
        (f_{n-1},...,f_1) &i = n.\end{cases}\]
    For $n\ge0$ and $0\le i\le n$, the degeneracy map $s_i : (NI)_n\to (NI)_{n+1}$ is given by
    \[s_i(f_n,...,f_1) = (f_n,...,f_{i+1},\id,f_i,...,f_1).\]
    \item[(b)] Let $J$ be another small category and $F : I\to J$ a functor. We define maps $(NF)_n : (NI)_n\to (NJ)_n$ by
    \[(NF)(f_n,...,f_1) = (F(f_n),...,F(f_1)).\]
    Shows that these maps form a morphism $NF : NI\to NJ$ of simplicial sets.
    
    \item[(c)] Show that the nerve construction preserves products, i.e., that the morphism
    \[
    N(I \times J) \xrightarrow{(N_{\operatorname{proj}_I} , N_{\operatorname{proj}_J} )}NI \times NJ
    \]
    is an isomorphism of simplicial sets.
    
    \item[(d)] Let $[1]$ denote the category with two objects 0 and 1 and three morphisms $\id_0, \id_1$ and $f : 0\to 1$. Let $\tau : F\to G$ be a natural transformation of functors from $I$ to $J$. Define a functor $H : I\times[1]\to J$ satisfying $H(-,0) = F$, $H(-,1) = G$ and $H(i,f) = \tau_i : F(i)\to G(i)$. Show that the natural transformation $\tau$ yields a simplicial homotopy between the morphisms $NF, NG : NI\to NJ$.

    \item[(e)] Show that the nerves $NI$ and $NJ$ of two equivalent small categories are homotopy equivalent simplicial sets.
\end{enumerate}

\begin{sketch}
\todo[inline,color=yellow]{To be added}
\end{sketch}

\subsection{AT1Sheet4}

The next exercise shows a way to construct the classifying space of a group $G$, a path-connected CW-complex $K(G,1)$ with fundamental group isomorphic to $G$ and all higher homotopy groups trivial (even though we do not prove all the relevant facts about it in the exercise). This is known as the bar construction.

\unnumpar{AT1Sheet4.1}\label{exercise:AT1Sheet4.1}
Let $G$ be a group. For $n\ge0$, let $(BG)_n = G^n$ be the cartesian product of $n$ copies of the underlying set of $G$. For $n\ge1$ and $0\le i\le n$ define $d_i : (BG)_n \to (BG)_{n-1}$ by
$(g_n,...,g_1) \mapsto (g_n,...,g_i,g_{i+1}\cdot g_i,g_{i-1},...,g_1)$. For $n\ge1$ and $0\le i\le n -1$ define
$s_i : (BG)_{n-1}\mapsto(BG)_n$ by
$s_i(g_{n-1},...,g_1) = (g_{n-1},...,g_{i+1},1,g_i,...,g_1)$.
\begin{enumerate}
    \item[(a)] Show that $BG$ extends to a simplicial set. Identify $BG$ as the nerve, in the sense of \hyperref[exercise:AT1Sheet3.3]{AT1Sheet3.3}, of a suitable category.
    \item[(b)] In the geometric realization $|BG|$ we take the class of $(e,1)\in(BG)_0 \times\nabla^0$ as the basepoint and call it `e'. Every element $g\in G = (BG)_1$ yields a continuous map
    \[\{g\}\times\nabla^1 \xhookrightarrow{\text{inclusion}} \bigcup_{n\ge0} (BG)_n \times\nabla^n \xrightarrow{\text{projection}} |BG|.\]
    Show that this map takes the two boundary points of $\{g\}\times\nabla^1$ to the basepoint of $|BG|$.
    \item[(c)] We identify the interval $[0,1]$ with $\{g\}\times\nabla^1$ via the homeomorphism sending t to $(g,(t,1 -t))$. By part (b) the composition
    \[[0,1]\xrightarrow{\cong}\{g\}\times\nabla^1 \to |BG|\]
    is a loop at the basepoint $e\in |BG|$. We let $w(g)$ denote the homotopy class of this loop in the fundamental group $\pi_1(|BG|,e)$. Show that
    \[w:G\to\pi_1(|BG|,e)\]
    is a group homomorphism.
\end{enumerate}

\begin{sketch}
\todo[inline,color=yellow]{To be added}
\end{sketch}

A cheap Poincaré conjecture.

\unnumpar{AT1Sheet4.2}\label{exercise:AT1Sheet4.2}
Show that every compact simply-connected 3-manifold without boundary
is homotopy equivalent to $S^3$. (Hint: you might want to use Poincaré duality and the
Hurewicz theorem.)

\begin{sketch}
\todo[inline,color=yellow]{To be added}
\end{sketch}

Mapping tori.

\unnumpar{AT1Sheet4.3}\label{exercise:AT1Sheet4.3}
To be added.

\begin{sketch}
\todo[inline,color=yellow]{To be added}
\end{sketch}

\subsection{AT1Sheet5}

Using the long exact sequence associated to a Serre fibration (theorem \ref{theorem:long-exact-sequence-serre-fibration}) we can compute some interesting homotopy groups.

\unnumpar{AT1Sheet5.1}\label{exercise:AT1Sheet5.1}
We denote by $V_{n}(\R^{k})$ the Stiefel manifold of $n$-frames in $\R^{k}$. An element of $V_{n}(\R^{k})$ is an $n$-tuple $(x_{1}, \ldots, x_{n})$ of orthonormal vectors from $\R^{k}$. The set $V_{n}(\R^{k})$ is topologized as a subspace of $(\R^{k})^{n}$. We denote by $Gr_n(\R^k)$ the Grassmann manifold of $n$-dimensional vector subspaces of $\R^{k}$. The set $Gr_n(\R^k)$ carries the quotient topology with respect to the map $q: V_n(\R^k) \rightarrow Gr_n(\R^k)$ that takes an $n$-frame onto its span. The complex Stiefel and Grassmann manifolds $V_{n}(\CC^{k})$ and $Gr_{n}(\CC^{k})$ are defined analogously. Show that the following maps are fiber bundles and identify the fibers:
\begin{equation*}
\begin{cases}
q: V_n(\R^k) \rightarrow Gr_n(\R^k) \quad \text{for } 1 \leq n \leq k, \\
q: V_{n}(\CC^{k}) \rightarrow Gr_{n}(\CC^{k}) \quad \text{for } 1 \leq n \leq k, \\
p: V_n(\R^k) \rightarrow V_{m}(\R^{k}) \quad \text{for } 1 \leq m < n \leq k, \\
p: V_{n}(\CC^{k}) \rightarrow V_{m}(\CC^{k}) \quad \text{for } 1 \leq m < n \leq k,
\end{cases}
\end{equation*}
here the maps $p$ forget the last $m-n$ vectors: $p(x_{1}, \ldots, x_{n}) = (x_{1}, \ldots, x_{m})$. Use the long exact homotopy group sequences to show that $V_n(\R^k)$ is $(k - n - 1)$-connected and $V_{n}(\CC^{k})$ is $(2k - 2n)$-connected. Calculate $\pi_{k-n}(V_n(\R^k))$ and $\pi_{2k - 2n + 1}(V_{n}(\CC^{k}))$.

\begin{sketch}
\todo[inline,color=yellow]{To be added}
\end{sketch}

\unnumpar{AT1Sheet5.2}\label{exercise:AT1Sheet5.2}
Let $G$ be a topological group that is also a Hausdorff space. Let $H$ be a subgroup of $G$ and $G/H$ the space of right cosets with the quotient topology. Show:
\begin{enumerate}
    \item[(a)] If $H$ is closed in $G$, then $G/H$ is a Hausdorff space.
    \item[(b)] Let $H$ be closed in $G$ and suppose that there is a \textit{local section} i.e., a neighbourhood $U$ if $1\cdot H$ in $G/H$ and a continuous section $\sigma: U \rightarrow G$ (i.e., $p \circ \sigma = \id_{U}$). Let $K$ be a closed subgroup of $H$. Then the projection
    \begin{equation*}
        G/K \rightarrow G/H, \quad gK \mapsto gH
    \end{equation*}
    is a fibre bundle with fibre $H/K$.
\end{enumerate}

\begin{sketch}
\todo[inline,color=yellow]{To be added}
\end{sketch}

\unnumpar{AT1Sheet5.3}\label{exercise:AT1Sheet5.3}
Let $p: E \rightarrow B$ be a fiber bundle with path connected base, $F = p^{-1}(b)$ the fiber over a point $b \in B$, and $x \in F$. Suppose that the inclusion $F \rightarrow E$ is homotopic to a constant map.

Show that the long exact homotopy group sequence degenerates into an isomorphism between $\pi_{n}(B, b)$ and $\pi_{n}(E, x) \times \pi_{n - 1}(F, x)$ for all $n \geq 1$. Apply this to the Hopf fibrations $\nu: S^{7} \rightarrow S^{4}$ and $\sigma: S^{15} \rightarrow S^{8}$ to deduce that the groups $\pi_{7}(S^{4}, z)$ and $\pi_{15}(S^{8}, z)$ each contain a copy of $\Z$ as a direct summand.

\begin{sketch}
\todo[inline,color=yellow]{To be added}
\end{sketch}

\subsection{AT1Sheet6}

\unnumpar{AT1Sheet6.1}\label{exercise:AT1Sheet6.1}
Let $p : S^m \to S^n$ be a fiber bundle with $m,n\ge1$ whose fiber is homeomorphic to the sphere $S^k$. Show that then $k=n-1$ and $m=2n-1$.

\begin{sketch}
\todo[inline,color=yellow]{To be added}
\end{sketch}

\unnumpar{AT1Sheet6.2}\label{exercise:AT1Sheet6.2}
Show by examples that various hypotheses in the exponential law are really necessary.
\begin{enumerate}
    \item[(a)] Find spaces $X$ and $Z$ such that the evaluation map
    \[
    \ev : Z^X\times X \to Z,\quad (f,x) \mapsto f(x)
    \]
    is not continuous.
    \item[(b)] Find spaces $X$, $Y$ and $Z$ such that the exponential map
    \[\Phi : Z^{X\times Y} \to (Z^X)^Y\]
    is not surjective.
    \item[(c)] Find spaces $X$, $Y$ and $Z$ such that $X$ is locally compact but the exponential map $\Phi$ is not a homeomorphism.
\end{enumerate}

\begin{sketch}
\todo[inline,color=yellow]{To be added}
\end{sketch}

\unnumpar{AT1Sheet6.3}\label{exercise:AT1Sheet6.3}
Let $X$ be a topological space with basepoint $x_0$. Let
\[
    E = \{f \in X^{\nabla^2}\ |\ f(1,0,0) = f(0,1,0) = f(0,0,1) = x_0\}
\]
be the space of continuous maps from the 2-simplex to $X$ that takes all three vertices to the basepoint. We define continuous maps
\(i,j : [0,1] \to\nabla^2\)
by \(i(t) = (t,1-t,0)\), respectively \(j(t) = (0,t,1-t)\).
Show that the map
\[\Phi: E \to\Omega X \times \Omega X,\quad f \mapsto (f\circ i,f\circ j)\]
is a homotopy equivalence.

\begin{sketch}
\todo[inline,color=yellow]{To be added}
\end{sketch}

\subsection{AT1Sheet7}

\unnumpar{AT1Sheet7.1}\label{exercise:AT1Sheet7.1}
Let $X$ be a topological space and $Z$ a Hausdorff space. Let $f: X \rightarrow Z$ be a continuous map and $\{f_{n}\}_{n \geq 1}$ a sequence in $Z^{X}$. Suppose that $f$ is a limit of the sequence $\{f_{n}\}$ in the compact-open topology on $Z^{X}$. Show that then the sequence $\{f_{n}\}$ converges pointwise to $f$. Show by example that the converse need not hold.

\begin{sketch}
\todo[inline,color=yellow]{To be added}
\end{sketch}

\unnumpar{AT1Sheet7.2}\label{exercise:AT1Sheet7.2}
Let $p : E\to B$ be a Serre fibration whose total space is contractible.
Show that the fiber of $p$ over a point $b\in B$ is weakly homotopy equivalent to the loop space of $B$, based at $b$.

\begin{sketch}
\todo[inline,color=yellow]{To be added}
\end{sketch}

\unnumpar{AT1Sheet7.3}\label{exercise:AT1Sheet7.3}
Given a permutation $\sigma \in \Sigma_3$ of the set $\{1,2,3\}$, define the homeomorphism
\[
    \sigma_* : \nabla^2 \to \nabla^2 \quad\text{by}\quad \sigma_*(x_1,x_2,x_3) = (x_{\sigma(1)},x_{\sigma(2)},x_{\sigma(3)}).
\]
For each of the six elements of the group $\Sigma_3$, identify the composite
{\small
\[
    \pi_1(X) \times\pi_1(X)\!\cong\!\pi_0(\Omega X \times\Omega X) \xrightarrow{\pi_0(\Phi)^{-1}}
    \pi_0(E) \xrightarrow{\pi_0(\sigma_*)}
    \pi_0(E) \xrightarrow{\pi_0(\Phi)}
    \pi_0(\Omega X \times\Omega X)\!\cong\!\pi_1(X) \times\pi_1(X)
\]}
explicitly in terms of group theoretic operations. Here $\Phi : E \to\Omega X \times\Omega X$ is the homotopy
equivalence introduced in \hyperref[exercise:AT1Sheet6.3]{AT1Sheet6.3}, and $\sigma_* : E \to E$ is the homeomorphism obtained
by restricting $(\sigma_*)^* : X^{\nabla^2}
\to X^{\nabla^2}$ to $E$.

\begin{sketch}
\todo[inline,color=yellow]{To be added}
\end{sketch}

\subsection{AT1Sheet8}

In \hyperref[exercise:AT1Sheet2.3]{AT1Sheet2.3} we studied a construction related to the homotopy colimit, now we study the homotopy limit.

\unnumpar{AT1Sheet8.1}\label{exercise:AT1Sheet8.1}
Let
\[
\cdots \to P_{n+1}
\xrightarrow{p_n} P_n
\xrightarrow{p_{n-1}} P_{n-1}
\xrightarrow{p_{n-2}} \cdots
\xrightarrow{p_0} P_0
\]
be a sequence of topological spaces and continuous maps. The homotopy limit of the
sequence is the space
\[
\holim_n P_n =
\left\{ (\omega_n)_{n\ge0} \in\prod_{n\ge0} P_n^{[0,1]}\ \Bigg|\ \omega_n(1) = p_n(\omega_{n+1}(0)) \text{ for all } n\ge0\right\}.
\]
The homotopy limit has the subspace topology of the product topology. We define continuous maps
\[
q_k : \holim_n P_n \to P_k \quad\text{ by }\quad q_k((\omega_n)_{n\ge0}) = \omega_k(0).\]
Let $\omega = (\omega_n)$ be a basepoint in $\holim_nP_n$ and let $m\ge1$.
\begin{enumerate}
    \item[(a)] We define a homomorphism $\Psi_n$ as the composite 
    \[
    \pi_m(P_{n+1},\omega_{n+1}(0))) \xrightarrow{(p_n)_*}
    \pi_m(P_n,p_n(\omega_{n+1}(0))) \xrightarrow{(\omega_n)_*} \pi_m(P_n,\omega_n(0)).
    \]
Here $(\omega_n)_*$ is the isomorphism given by conjugation with the path $\omega_n$. Show that
the homomorphisms
    \[
    \pi_m(q_k) : \pi_m(\holim_nP_n,\omega) \to \pi_m(P_k,\omega_k(0))
    \]
satisfy $\Psi_k \circ \pi_m(q_{k+1}) = \pi_m(q_k)$
and hence they assemble into a homomorphism
\begin{equation}
\pi_m(\holim_nP_n,\omega) \to \lim_n \pi_m(P_n,\omega_n(0)),
\end{equation}
where the limit in the right hand side is taken over the homomorphisms $\Psi_n$.
\item[(b)] Show the homomorphism (1) defined in (a) is surjective.
\item[(c)] Suppose that there is an $N\ge0$ such that for all $n\ge N$ the map
\[
\Psi_n : \pi_{m+1}(P_{n+1},\omega_{n+1}(0)) \to \pi_{m+1}(P_n,\omega_n(0))
\]
is surjective. Show that then the homomorphism (1) is bijective.
\end{enumerate}

\begin{sketch}
\todo[inline,color=yellow]{To be added}
\end{sketch}

\unnumpar{AT1Sheet8.2}\label{exercise:AT1Sheet8.2}
Prove the following.
\begin{enumerate}
    \item[(a)] Let
    \begin{center}
    \begin{tikzcd}
G_{0} \arrow[r, "\alpha_{0}"] & G_{1} \arrow[r, "\alpha_{1}"] & \ldots \arrow[r, "\alpha_{m-1}"] & G_{m} \arrow[r, "\alpha_{m}"] & \ldots
\end{tikzcd}
\end{center}
be a sequence of groups and groups homomorphisms. Consider $n \geq 1$; if $n \geq 2$ assume that all groups $G_{m}$ are abelian. Let $X_{m}$ be an Eilenberg-Maclane space of type $K(G_{m}, n)$ and $f_{m}: X_{m} \rightarrow X_{m + 1}$ a continuous based map that realizes $\alpha_{m}$ on $\pi_{n}$. Show that the mapping telescope of the sequence $\{f_{m}\}$ is an Eilenberg-Maclane space of type $K(G_{\infty}, n)$, where $G_{\infty}$ is a colimit of the original sequence of group homomorphisms.
    \item[(b)] Let $f: S^{1} \rightarrow S^{1}$ be the standard degree $n$ map on the circle defined by $f(z) = z^{n}$. Show that the mapping telescope of the sequence
    \begin{center}
    \begin{tikzcd}
S^{1} \arrow[r, "f"] & S^{1} \arrow[r, "f"] & \ldots \arrow[r, "f"] & S^{1} \arrow[r, "f"] & \ldots
\end{tikzcd}
    \end{center}
    is an Eilenberg-Maclane space and describe its fundamental group.
\end{enumerate}

\begin{sketch}
\todo[inline,color=yellow]{To be added}
\end{sketch}

\unnumpar{AT1Sheet8.3}\label{exercise:AT1Sheet8.3}
Let $A$ be an abelian group and $n\ge2$. Show that the homology group
$H_{n+1}(K(A,n),\Z)$
is trivial. (Hint: construct an Eilenberg-MacLane space from a \tbf{Moore space}, i.e. a space with only one prescribed non-trivial reduced homology group.)

\begin{sketch}
\todo[inline,color=yellow]{To be added}
\end{sketch}

\subsection{AT1Sheet9}

\unnumpar{AT1Sheet9.1}\label{exercise:AT1Sheet9.1}
To be added.

\begin{sketch}
\todo[inline,color=yellow]{To be added}
\end{sketch}

\unnumpar{AT1Sheet9.2}\label{exercise:AT1Sheet9.2}
To be added.

\begin{sketch}
\todo[inline,color=yellow]{To be added}
\end{sketch}

The next exercise shows a way of constructing a topological space with prescribed homotopy groups in each dimension.

\unnumpar{AT1Sheet9.3}\label{exercise:AT1Sheet9.3}
To be added.

\begin{sketch}
\todo[inline,color=yellow]{To be added}
\end{sketch}

\subsection{AT1Sheet10}

\unnumpar{AT1Sheet10.1}\label{exercise:AT1Sheet10.1}
Show that every CW-complex that is $n$-connected and $n$-dimensional is contractible.

\begin{sketch}
\todo[inline,color=yellow]{To be added}
\end{sketch}

\unnumpar{AT1Sheet10.2}\label{exercise:AT1Sheet10.2}
Show that for every continuous map $f : X \to Y$, the following conditions
are equivalent.
\begin{enumerate}
\item[(i)] The map $f$ is a weak homotopy equivalence.
\item[(ii)] For all $n \ge0$ and all continuous maps $\alpha : \partial D^n \to X$ and $\beta : D^n \to Y$ such that
$\left.\beta\right|_{\partial D^n} = f \circ\alpha$, there is a continuous map $\lambda : D^n \to X$ such that $\left.\lambda\right|_{\partial D^n} = \alpha$ and such
that $f\circ \lambda : D^n \to Y$ is homotopic, relative to $\partial D^n$, to $\beta$.
\item[(iii)] For every CW-complex $K$ and every subcomplex $L$ of $K$, all continuous maps $\alpha :
L \to X$ and $\beta : K \to Y$ such that $\left.\beta\right|_L = f\circ\alpha$, there is a continuous map $\lambda : K \to X$
such that $\left.\lambda\right|_L = \alpha$ and such that $f\circ \lambda : K \to Y$ is homotopic, relative to $L$, to $\beta$.
\item[(iv)] For every CW-complex $K$, the induced map
$[K, f]: [K, X] \to [K, Y]$
of homotopy classes of continuous maps is bijective.
\end{enumerate}

\begin{sketch}
\todo[inline,color=yellow]{To be added}
\end{sketch}

\unnumpar{AT1Sheet10.3}\label{exercise:AT1Sheet10.3}
Let $A$ and $B$ be abelian groups, and $n\ge1$.
\begin{enumerate}
    \item[(i)] Show that for $1\le m<n$ the only natural transformation $H^n(X; A) \to H^m(X; B)$ is the zero transformation.
    \item[(ii)] Every group homomorphism $\varphi : A \to B$ gives rise to a coefficient homomorphism
    \[
    \varphi_* : H^n(X; A) \to H^n(X; B)
    \]
    where $X$ is any space. Show that this assignment defines an isomorphism of groups
    \[
    \Hom(A, B) \to \Nat(H^n(-; A), H^n(-; B))
    \]
    to the group of cohomology operations of type $(A, n, B, n)$.
    \item[(iii)] In an earlier exercise we constructed from a short exact sequence of abelian groups
    \[ 0 \to B \xrightarrow{i} E \xrightarrow{p} A \to 0\]
    a Bockstein homomorphism
    \[
    \beta(i, p): H^n(X; A) \to H^{n+1}(X; B)
    \]
    where $X$ is any space. The Bockstein homomorphism only depends on the class of the extension in $\Ext(A, B)$.
    Show that this assignment defines an isomorphism of groups
    \[
    \beta : \Ext(A, B) \to \Nat(H^n(-; A), H^{n+1}(-; B))
    \]
    to the group of cohomology operations of type $(A, n, B, n + 1)$.
\end{enumerate}

\begin{sketch}
\todo[inline,color=yellow]{To be added}
\end{sketch}

\subsection{AT1Sheet11}

\unnumpar{AT1Sheet11.1}\label{exercise:AT1Sheet11.1}
Prove the following.
\begin{enumerate}
    \item[(i)]A simplicial set $X$ is $m$-dimensional if all simplices of $X_n$ for $n > m$ are degenerate.
Show that the product of an $m$-dimensional simplicial set and an $n$-dimensional
simplicial set is $(m + n)$-dimensional.
    \item[(ii)] Identify the non-degenerate simplices of the simplicial set $\Delta^n \times\Delta^1$. How many non-degenerate $(n + 1)$-simplices does $\Delta^n\times\Delta^1$ have?
    \end{enumerate}

\begin{sketch}
\todo[inline,color=yellow]{To be added}
\end{sketch}

The next exercise is an important generalization of \hyperref[exercise:AT1Sheet3.2]{AT1Sheet3.2}, which will be used often throughout chapter \ref{chapter:the-cool-chapter}.

\unnumpar{AT1Sheet11.2}\label{exercise:AT1Sheet11.2}
Prove the following.
\begin{itemize}

    \item[(i)] Let $K$ be a compact space. Show that for every space $X$ the map
    \[\eta: X\to (X\times K)^K,\ \eta(x)(k)=(x,k)\]
    is continuous.
    
    \item[(ii)] Show that for every compact space $K$, the functor $-\times K$ from $\Top$ to itself is left adjoint to the functor sending a space $Y$ to the space $Y^K$ with the compact-open topology.
    
    \item[(iii)] Show that for every simplicial set A the map
    \[(|p_1|,|p_2|):|X\times\Delta^1|\to|X|\times|\Delta^1|\]
    is a homeomorphism, where where $p_1:A\times\Delta^1\to A$ and $p_2:A\times\Delta^1\to \Delta^1$ are the projections to the two factors. (Hint: use the simplicial skeleton filtration for A and the fact that $-\times|\Delta^1|$ preserves colimits to reduce the claim to the special case A = $\Delta^n$ which was shown earlier.)
    
\end{itemize}

\begin{sketch}
\todo[inline,color=yellow]{To be added}
\end{sketch}

This is essentially the proof of proposition \ref{theorem:classifying space}.

\unnumpar{AT1Sheet11.3}\label{exercise:AT1Sheet11.3}
Let $G$ be a group, and let $X$ be a $G$-simplicial set. Suppose that the $G$-action on the set $X_0$ of vertices is free.
\begin{itemize}
    \item[(i)] Show that the action of $G$ on the set $X_n$ of $n$-simplices is free for every $n\geq0$.
    \item[(ii)] Show that the action of $G$ on the geometric realization $|X|$ is free and properly discontinuous, i.e. every point $x\in|X|$ has a neighborhood in $U$ in $X$ such that $U\cap(g\cdot U)=\emptyset$ for all $g\in G$ with $g\neq 1$.
\end{itemize}

\begin{sketch}
\todo[inline,color=yellow]{To be added}
\end{sketch}

\subsection{AT1Sheet12}

\todo[inline,color=yellow]{To be added}
