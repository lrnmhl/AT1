% Lecture 26

\lecture[Cooler stuff (mostly without proof).]{2022-01-26}

Recall: a simplicial set is a Kan complex if all its horns have fillers, a quasi-category if all its inner horns have fillers.

As an example, last time we saw that the nerve of every (usual) category has unique fillers of all inner horns.

Let $f,g:x\to Y$ be morphisms in a quasi-category. Then $f$ and $g$ are homotopic if the following four conditions hold:

\[some\ diagram\]

Homotopy is an equivalence relation on the set of morphisms from $x$ to $y$.

A composite of... [something here]

Composition is well-defined on homotopy classes... [something here]

\begin{thmdef}
Let $\Cc$ be a quasi-category. The \tbf{homotopy category} $h\Cc$ has objects $\Cc_0$, morphisms the homotopy classes of $\Cc$-morphisms and composition as above.
\end{thmdef}

\begin{proof}
The identity property is clear.

We want to show that composition is associative: let
\[x\xto{f}y\xto{g}z\xto{i}w\]
be composable $\Cc$-morphisms. Let $\sigma,\tau\in\Cc_2$ be $2$-simplices with boundaries $\de\sigma=(g,h,f)$ (read \enquote{$h\simeq g\circ f$}) and $\de\tau(i,h,g)$. Then choose $\alpha\in\Cc_2$ with $\de\alpha=(i,j,h)$. This data provides a $\Lambda^3_2$ horn:
\[diagram\]
We choose a filler $\mu\in\Cc_3$ of this horn and consider $d_2^*(\mu)\in\Cc_2$. Then
\[[i]\circ([g]\circ[f])=[i]\circ[h]=[j]=[k]\circ[f]=([i]\circ[g])\circ[f]\]
where the penultimate equality is given by $d_2^*(\mu)$.
\end{proof}

\begin{example}
Let $\Cc=NC$ be the nerve of some category $C$. Then $f,g:x\to y$ are homotopic if and only if [something].

Hence in $NC$ \enquote{homotopy} specializes to \enquote{equality}. So $h(NC)=C$.
\end{example}

\begin{theorem}[Joyal]
A quasi-category $\Cc$ is a Kan-complex if and only if $h\Cc$ is a groupoid.
\end{theorem}

\begin{example}
There are essentially two ways of obtaining quasi-categories, from old ones (many constructions, like completion or cocompletition, slice categories et cetera, specialize to quasi-categories) or from modified nerve constructions. We focus now on the second way.
\[*****\]
Let $\Cc$ be a $2$-category. The \tbf{Duskin nerve} $NC$ is given by:
\begin{itemize}
    \item $(NC)_0$ are the objects of $C$,
    \item $(NC)_1$ are the $1$-simplices of $C$,
    \item $(NC)_2$ are quadruples:
    \[diagram\]
    \item $(NC)_3$:
    \[not\ precise\ complicated\ diagram\]
    such that the following two $2$-cells are equal
    \[stuff\]
\end{itemize}
\end{example}

\begin{theorem}
The nerve of a $(2,1)$-category ($2$-category with invertible $2$-cells) is a quasi-category.
\end{theorem}

\begin{remark}
A category $C$ can be made into a $(2,1)$-category with only identity $2$-cells. In this case the Duskin nerve specializes to the ordinary nerve
\end{remark}

\begin{example}
Categories can be considered as $1$-categories (ignoring natural transformations) and they form a category $\Cat_1$, or as $2$-categories and they form a category $\Cat_2$.
We have
\[hN(\Cat_1)=\Cat_1\]
hence two categories are isomorphic if and only if they are isomorphic in the usual sense.
But in
\[h(N^\text{Duskin}(\Cat_{(2,1)}))\]
two categories are isomorphic if and only if they are equivalent in the usual sense.
\end{example}

Let $C$ be a category enriched in Kan complexes, i.e. a set of objects $\ob(C)$, Kan complexes of morphisms $\map_C(x,y)$ for all $x,y\in\ob(C)$, composition of morphisms
\[\circ:\map_C(y,z)\times\map_C(x,y)\to\map_C(x,z),\]
identities $1_x\in\map_C(x,x)_0$, that is (strictly) associative and unital.

\begin{example}
The category $\Kan$: objects are all Kan complexes,
\[\map_\Kan(X,Y)=\map(X,Y)\]
where $\map(X,Y)$ is the mapping simplicial set, with simplices
\[\map(X,Y)_n=\Hom_\sSet(X\times\Delta^n,Y).\]
If $Y$ is Kan, so is $\map(X,Y)$.
\end{example}

The \tbf{coherent nerve} of a category $C$ enriched in Kan complexes is:
\begin{itemize}
    \item $(NC)_0$ are the objects of $C$,
    \item $(NC)_1$ is the set of $1$-simplices in all mapping complexes,
    \[\Hom_{NC}(x,y)=\map_C(x,y)_1,\]
    \item $(NC)_2$ is the set of triples $(f,g,h,\tau)$...
    \[something\]
    [something more]
\end{itemize}

\begin{proposition}
Let $K$ and $\Cc$ be simplicial sets.
\begin{numerate}
\item If $\Cc$ is a Kan complex, then so is $\map(K,\Cc)$.
\item If $\Cc$ is a quasi-category, then so is $\map(K,\Cc)$, the quasi-category of functors $K\to\Cc$.
\end{numerate}
\end{proposition}

The $\infty$-category of spaces is the quasi-category $N^\text{coh}(\Kan)$ and $hN^\text{coh}(\Kan)=\Ho(\sSet_\text{Kan})$.

The \tbf{core} of a quasi-category $\Cc$ is the simplicial subset of $\Cc$ with
\[(\core\Cc)_n=\cb{\sigma\in\Cc_n:\text{all edges of }\sigma\text{ are equivalent}}.\]

\begin{theorem}
$\core\Cc$ is a quasi-category, and in fact a Kan complex.
\end{theorem}

The $\infty$-category of $\infty$-categories is the coherent nerve of the Kan enriched category with objects all quasi-categories and morphisms simplicial sets $\core\map(\Cc,\Dd)$. Then $\ob(\Cat_\infty)$ are all quasi-categories, $\map_{\Cat_\infty}(\Cc,\Dd)$ are functors $\Cc\to\Dd$... [something]

A functor $f:\Cc\to\Dd$ of quasi-categories is an equivalence if for every simplicial set $K$, the functor
\[f_*:h\map(K,\Cc)\to h\map(K,\Dd)\]
is an equivalence of ($1$-)categories.

Note: if $K=*$, $\map(*,\Cc)\cong\Cc$, then $f_*:h\Cc\to h\Dd$ is an equivalence. But equivalence of homotopy categories is not sufficient: the projection
\begin{align*}
    \Cc&\to N(h\Cc)\\
    x&\mapsto x\\
    x\xto{f}y&\mapsto[f]
\end{align*}
induces an equivalence of homotopy categories. But this is typically \tit{not} an equivalence of categories.

A functor $f:\Cc\to\Dd$ of quasi-categories is a \tbf{localization} of a set $\Ww$ of $\Cc$-morphisms if for all quasi-categories $\Dd$, the resulting functor
\[\gamma:\Fun(\Cc_\Ww,\Dd)\to\Fun(\Cc,\Dd)\]
is an equivalence (of quasi-categories!) onto the full subcategory of functors that send $\Ww$ to equivalences. 

Localization of quasi-categories exist and are unique up to equivalences under $\Cc$.

\begin{theorem}
The following quasi-categories are equivalent:
\begin{itemize}[label={-}]
    \item the $\infty$-category of spaces $N^\text{coh}(\Kan)$,
    \item the localization of $N(\sSet)$ at the weak equivalences,
    \item the localization of $N(\Top)$ at the weak homotopy equivalences,
    \item $N^\text{coh}(\Top_\text{CW})$ Kan-enriched by $\map(X,Y)=\S(\map(X,Y))$.
\end{itemize}
\end{theorem}
