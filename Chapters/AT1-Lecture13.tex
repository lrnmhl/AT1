% Lecture 13

\lecture[We introduce Eilenberg-MacLane spaces and classifying spaces.]{2021-11-29}

There's one last important easy example of EM-space.

\begin{example}
For $n\geq2$, we have $\Omega K(A,n)=K(A,n-1)$.\rightnote{Sometimes the notation with EM-spaces gets a bit sloppy.}
\end{example}

We want to show existence and uniqueness (up to homotopy) of EM-spaces. For the existence part, it is better to treat the dimension 1 case separately (the reason is that in this case we do not have Hurewicz's theorem).

\section{Construction of Classifying Spaces.}
\label{section:construction-of-classifying-spaces}

A space of type $(G,1)$ is also denoted $BG$, and called the \tbf{classifying space} for $G$.

We use a construction that is similar to the one we saw in AT1Sheet4-1.

\begin{construction}[Bar construction]
Let $X$ be a set. We define a simplicial set $EX$ by setting $(EX)_n=\Hom(\cb{0,1,\dots,n},X)(\cong X^{n+1}$ via $f\mapsto(f(0),f(1),\dots,f(n))$) where the simplicial structure map $\alpha^*:(EX)_n\to(EX)_m$ induced by $\alpha:[m]\to[n]$ is $\alpha^*(f)=f\circ\alpha$. Equivalently we have: $\alpha^*(x_0,\dots,x_n)=(x_{\alpha(0)},\dots,x_{\alpha(m)})$.

If $g:X\to Y$ is any map, a morphism of simplicial sets $g_*:EX\to EY$ is given by
\[g_*(f)=g\circ f\text{ or }g_*(\xso)=(g(x_0),\dots,g(x_n)).\]

Altogether we get a functor $E:\Set\to\sSet$.
\end{construction}

\begin{proposition}
If $X$ is not empty, then $EX$ is simplicially contractible.
\end{proposition}

\begin{proof}
Pick any element $y\in X$. We will write down an explicit morphism of simplicial sets
\[H:EX\times\Delta[1]\to EX\]
that contracts $EX$ onto $y$.

For $0 \le i \le n+1$, let $k_i:[n]\to[1]$ be the weakly monotone map with
\[k_i(i-1)=0\text{ and }k_i(i)=1.\]
We define $H_n:(EX)_n\times\Delta[1]_n=X^{n+1}\times\Delta([n],[1])\to X^{n+1}$ as
\[H_n((\xso),k_i)=(x_0,\dots,x_{i-1},y,\dots,y).\]

It is straightforward to check that these maps do form a morphism of simplicial sets and we have:
\[H_n((\xso),\const_0)=H_n((\xso),k_{n+1})=(\xso),\]
\[H_n((\xso),\const_1)=H_n((\xso),k_0)=(y,\dots,y).\]
\end{proof}

Now let $G$ be any group. Then $G$ acts on itself by right translation. So for $g\in G$ we get a morphism $E_{rg}:EG\to EG$, where $rg$ is right multiplication by $g$. This defines an action of $G$ on $EG$ by morphisms of simplicial set, i.e. it makes $EG$ into a $G$-simplicial set.

A \textbf{$G$-simplicial set} is a functor $Y:\Delta^\op\to G\text{-}\Set$. More explicitly, a $G$-simplicial set is the data of an underlying simplicial set $Y$ and a $G$-action on $Y_n$ for any $n\geq0$ such that $\alpha^*:Y_n\to Y_m$ is $G$-equivariant for all $\alpha:[m]\to[n]$ in $\Delta$.

The \textbf{simplicial orbit set} is the composite functor:
\[Y/G:\Delta^\op\xto{Y}G\text{-Set}\xto{-/G}\Set,\]
i.e. $(Y/G)_n=Y_n/G$, the set of $G$-orbits on $Y_n$.
\[\alpha^*:Y_n/G=(Y/G)_n\to(Y/G)_m=Y_m/G,\ yG\mapsto\alpha^*(y)G\]

This comes with a morphism of simplicial sets $Y\to Y/G$ with $p_n:Y_n\to Y_n/G$, $y\mapsto yG$.

\begin{theorem}
Let $G$ be a group and $Y$ a free $G$-simplicial set, i.e. the $G$ action on $Y_n$ is free for all $n\geq0$. Then $G$ acts freely and properly discontinuously on $|Y|$ and the continuous map $|p|:|Y|\to|Y/G|$ is a covering space with deck transformation group $G$.
\end{theorem}

Note: we did not prove the theorem, but the key part should be that $|-|:\Set\to\Top$ is a left adjoint, so it preserves all colimits, giving (somehow) that $|Y|/G\cong|Y/G|$.\todo{Not sure about what he was trying to say...\\ I guess he put it into the Exercise 11.3} If $G$ happens to be finite, we only need the freeness of the action on $|Y|$ and the fact that $|Y|$ is Hausdorff.

We apply this theorem to $Y=EG$. The right translation action of $G$ on itself is free, so the map:
\[|p|:|EG|\to|(EG)/G|\cong|EG|/G\]
is a covering space with deck transformation group $G$. Since $|EG|$ is contractible $|(EG)/G|$ is a $K(G,1)$.

\begin{remark}
The \textbf{bar-construction} $BG$ of a group $G$ (introduced in AT1Sheet 4) is the simplicial set $(BG)_n=G^n$ with structure maps:
\[d^*_i(g_1,\dots,g_n)=\begin{cases}
(g_2,\dots,g_n) & i=0\\
(g_1,\dots,g_ig_{i+1},\dots,g_n) & 1\leq i\leq n-1\\
(g_1,\dots,g_\ni)
\end{cases}\]
\[s^*_i(g_1,\dots,g_n)=(g_1,\dots,g_{i-1},1,g_{i},\dots,g_n).\]

\begin{lemma}
The simplicial sets $EG/G$ and $BG$ are isomorphic. Therefore $|BG|$ is homeomorphic to $|EG/G|$, hence a $K(G,1)$.
\end{lemma}

\begin{proof}
Consider $q:EG\to BG$ defined in dimension $n$ by:
\begin{align*}
    q_n:(EG)_n=G^{n+1}&\to G^n=(BG)_n\\
    (\nno{g}{n})&\mapsto(g_0g_1^{-1},g_1g_2^{-1},\dots,g_\ni g_n^{-1}).
\end{align*}
The map $q_n$ is constant on $G$-orbits by the composite right translation action. Then $q_n$ factors through a bijection $(EG)_n/G\xto{\cong}(BG)_n.$
\end{proof}
\end{remark}

We also have explicit isomorphism (see AT1Sheet4-1):
\begin{align*}
    G&\to\pi_1(|BG|,1)\\
    g&\mapsto\cb{[0,1]\to|BG|}\\
    t&\mapsto[(g,(t,1-t))]\cont G\times\sx{1}\to|BG|.
\end{align*}

\tbf{What do classifying spaces classify?} --- A short digression on what the classifying space $BG=K(G,1)$ actually classifies. Let $G$ be a group. A \textbf{$G$-principal bundle} is a $G$-space $X$ with the following property: for every $xG\in X/G$ there is an open neighbourhood $U$ of $xG$ in $X/G$ and a $G$-equivariant homeomorphism $p^{-1}(U)\to U\times G$ (making the usual diagram commute) where $p:X\to X/G$ is the quotient map.

\begin{example}
$|EG|$ with the $G$ action is a principal $G$-bundle.
\end{example}

\begin{itemize}[label={-}]
    \item The action of principal $G$-bundles are in particular free.
    \item Principal $G$-bundles admit base change. Consider a principal $G$-bundle and a continuous map $f:Y\to X/G$:
    \[\begin{tikzcd} & X \ar[d,"p"] \\
    Y \ar[r,"f"] & X/G\end{tikzcd}\]
    Then $Y\times_{X/G}X$ comes with a continuous $G$-action by $(y,x)g=(y,xg)$, i.e. $Y\times_{X/G}X$ is another principal $G$-bundle with \[(Y\times_{X/G}X)/G\xto{\cong} Y.\]
\end{itemize}

\begin{theorem}
For all paracompact spaces $Y$ and all groups $G$, the map:
\begin{align*}
    [Y,|BG|]&\to \operatorname{Prin}_G(Y)\\
    [f]&\mapsto[f^*(|EG|\to\underbrace{|EG/G|}_{=|BG|})]
\end{align*}
is a bijection.
\end{theorem}

\section{Construction of EM-Spaces}

The first step towards construction of EM-spaces is to have a way to "kill homotopy groups".

\begin{theorem}[Killing homotopy groups]\label{theorem:killing-homotopy-groups}
Let $n\geq0$ and let $(Y,y)$ be a based space. Then there is a relative CW-complex $(X,Y)$ such that:
\begin{itemize}
    \item[(i)] all relative cells have dimension $\geq n+1$,
    \item[(ii)] The inclusion induces isomorphisms $\pi_i(Y,y)\to\pi_i(X,y)$ for all $0\leq i<n$,
    \item[(iii)] For all $i\geq n$, $\pi_i(X,y)=0$.
\end{itemize}
\end{theorem}

\begin{proof}
We set $X^{(0)}=X^{(1)}=\cdots=X^{(n)}=Y$ and we construct the higher cells by induction, so that:
\begin{itemize}
    \item[(a)] $X^{(i+1)}$ is obtained from $X^{(i)}$ by attaching $(i+1)$-cells; in particular, the inclusion $X^{(i)}\into X^{(i+1)}$ induce isomorphisms of homotopy groups up to dimension $i-1$ (by cellar approximation),
    \item[(b)] the homotopy groups $\pi_i(X^{(i+1)},y)$ is trivial.
\end{itemize}

Then $X=\cup_{i\geq0}X^{(i)}$ with the weak topology is the desired CW-complex relative to $Y$. Then we have $\pi_i(X^{(N)},y)\cong\pi_i(X,y)$ for all large enough $N$, in particular $\pi_i(X,y)\cong 0$ for $i\geq n$ and $\pi_i(X,y)\cong\pi_i(Y,y)$ for $i<n$.

We describe the inductive procedure. Suppose that $X^{(i)}$ has already been constructed. Choose generators $\cb{x_j}_{j\in J}$ for the group $\pi_i(X^{(i)},y)$, choose representatives $f_j:S^i\to X^{(i)}$ of the classes $x_j$.

Define $X^{(i+1)}=X^{(i)}\cup_{S^i\times J}D^{i+1}\times J$ using the maps $f_j$ as attaching maps.

We observe:
\begin{enumerate}
    \item The map $\pi_i(X^{(i)},y)\to\pi_i(X^{(i+1)},y)$ is surjective by cellular approximation.
    \item The map $\pi_i(X^{(i)},y)\to\pi_i(X^{(i+1)},y)$ is the zero map; it suffices to show that the set $\cb{x_j}_{j\in J}$ of generators goes to zero. For all $j\in J$ the composite
    \[S^i\xto{f_j}X^{(i)}\into X^{(i+1)}\]
    extends to a continuous map on $D^{i+1}$, so it represents the trivial class in $\pi_i(X^{(i+1)},y)$.
\end{enumerate}

From this we get $\pi_i(X^{(i+1)},y)=0$.
\end{proof}
