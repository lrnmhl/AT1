% Lecture 14

\lecture[Existence of EM-spaces (btw, today is supposedly Dies Academicus...).]{2021-12-1}

We can now prove the existence of EM-spaces.\rightnote{I missed this lecture because I thought on the Dies Academicus there were no lectures. Apparently the Dies Academicus starts at 10, though. -.-}

\begin{theorem}
Let $n\geq2$ and let $A$ be an abelian group. Then there is an EM-space of type $(A,n)$ that is a CW-complex.
\end{theorem}

\begin{proof}
We choose a free resolution of $A$ as an abelian group:
\[0\to\Z[I]\xto{d}\Z[J]\xto{\epsilon}A\to0,\]
i.e. a short exact sequence of abelian groups with $I,J$ some sets.

We define a CW-complex with skeleta:
\[X^{(0)}=\cdots=X^{(\ni)}=\cb{x},\]
\[X^{(n)}=\cb{x}\cup_{J\times S^\ni}J\times D^n\cong\bigvee_J S^n.\]
By cellular approximation $X^{(n)}$ is $(n-1)$-connected. The Hurewicz theorem then provides an isomorphism
\[h:\pi_n(X^{(n)},x)\xto{\cong}H_n(X^{(n)};\Z)\cong\Z[J].\]

For every index $i\in I$ we chose a map $\alpha_i:S^n\to X^{(n)}$ such that $h([\alpha_i])=d(i)\in\Z[J]$. We define
\[X^{(n+1)}=X^{(n)}\cup_{I\times S^n}I\times D^{n+1}\]
using the $\alpha_i$'s as attaching maps. Then $X^{(n+1)}$ is a CW-complex with one $0$-cell, an $n$-cell for every element of $J$ and an $(n+1)$-cell for every element of $I$. So the cellular chain complex of $X^{(n+1)}$ is concentrated in degrees $0$, $n$ and $n+1$ and in the relevant dimension it looks as follows:
{\small
\[0\to C^\text{cell}_{n+1}(X^{(n+1)};\Z)=\underbrace{H_{n+1}(X^{(n+1)},X^{(n)};\Z)}_{\cong\Z[I]}\xto{\de}C^\text{cell}_n(X^{(n+1)};\Z)=\underbrace{H_n(X^{(n)},\cb{x};\Z)}_{\cong\Z[J]}\to0.\]}Hence we have $H_n^\text{cell}(X^{(n+1)};\Z)=\coker(\de)\cong\coker(d:\Z[I]\to\Z[J])\cong A$.

$X^{(n+1)}$ is again $(n-1)$-connected, so the Hurewicz theorem provides an isomorphism
\[\pi_n(X^{(n+1)},x)\xto{\cong}H_n(X^{(n+1)};\Z)\cong H^\text{cell}_n(X^{(n+1)};\Z)\cong A.\]

Now we can use kill the homotopy groups as we have seen in Theorem \ref{theorem:killing-homotopy-groups}, obtaining a relative CW-complex $(X,X^{(n+1)})$ with relative cells in dimensions $n+2$ and higher and such that:
\[\pi_i(X,x)\cong\pi_i(X^{(n+1)},x)\cong\begin{cases}
0 & \text{for } 1\le i < n\\
A & \text{for } i = n
\end{cases}\]
and $\pi_i(X,x)=0$ for $i\ge n+1$. So $X$ is a CW-complex and a $K(A,n)$.
\end{proof}

There is also an alternative construction of $K(A,n)$ as the geometric realization of some simplicial set.\todo{I am missing this part but I plan to add details eventually.}

\section{Uniqueness of EM-Spaces}

We first prove an auxiliary lemma.

\begin{lemma}\label{theorem:extension-theorem}
Let $(X,Y)$ be a relative CW-complex, $Z$ any space. Suppose that for all $m\geq1$ such that $(X,Y)$ has at least one relative $m$-cell, $\pi_{m-1}(Z,z)=0$ holds for all $z\in Z$. Then every continuous map $f:Y\to Z$ has a continuous extension to $X$.
\end{lemma}

\begin{proof}
We construct inductively continuous maps $f^{(n)}:X^{(n)}\to Z$ on the relative skeleta, that successively extend each other. Then $g=\cup f^{(n)}:X=\cup X^{(n)}\to Z$ does the job.\rightnote{Just to be sure:\\ do not forget that maps constructed like this are continuous thanks to CW-complexes having the final topology with respect to the inclusions of\\ their skeleta.}

We start with $f=f^{(-1)}:X^{(-1)}=Y\to Z$. We extend $f^{(-1)}$ to $f^{(0)}:X^{(0)}=Y\amalg J_0\to Z$ by mapping the relative $0$-cells arbitrarily to $Z$.

Let now $m\ge1$ and suppose that $f^{(m-1)}$ has been constructed. If there are no relative $m$-cells, we set $f^{(m)}=f^{(m-1)}$. Otherwise choose characteristic maps for the relative $m$-cells, call $\alpha_j$ the attaching maps and consider the following composite:
\[S^{m-1}\xto{\alpha_j}X^{(m-1)}\xto{ f^{(m-1)}}Z.\]
Pick a basepoint $x\in S^{m-1}$. Then $f^{(m-1)}\circ\alpha_j$ represents an element in $\pi_{m-1}(Z,z)$, with $z=f^{(m-1)}(\alpha_j(x))$. By hypothesis we have $\pi_{m-1}(Z,z)=0$, hence the $f^{(m-1)}\circ\alpha_j$ all represent the trivial class, i.e. they can be extended to the disk $D^m$, which gives us a way to extend $f^{(m-1)}$ to a map $f^{(m)}$ defined on $X^{(m)}$.
\end{proof}

\begin{theorem}\label{theorem:single-0-cell-complex}
Every $(\ni)$-connected CW-complex $Y$ is homotopy equivalent to a CW-complex whose $(\ni)$-skeleton is one $0$-cell.
\end{theorem}

\begin{proof}
Suppose that $n=1$. We can choose a maximal tree\rightnote{Strictly speaking, we would have to show that such a maximal tree exists, but heh.} in $Y$ (i.e. a 1-dimensional subcomplex containing all the $0$-cells) and collapse it to obtain the desired CW-complex with just one cell (the quotient map will be an homotopy equivalence).

Now suppose that $n\ge2$. Because $Y$ is $(n-1)$-connected, its $n$-skeleton is still $(n-1)$-connected. Since the Hurewicz map is an isomorphism, considering
\[\pi_n(Y^{(n)},y)\to H_n(Y^{(n)};\Z)\cong H^\text{cell}_n(Y^{(n)};\Z)=\ker(\underbrace{C^\text{cell}_n(Y^{(n)};\Z)}_{\text{free abelian}}\xto{d^\text{cell}}C^\text{cell}_\ni(Y^{(n)};\Z))\]
we have that $H_n(Y^{(n)};\Z)$ and hence $\pi_n(Y^{(n)},y)$ are free abelian groups.

We choose a basis $I$ of the free abelian group $\pi_n(Y^{(n)},y)$ and represent the basis elements by continuous maps $\alpha_i:S^n\to Y^{(n)}$. These $\alpha_i$'s together define a map
\[\alpha=\bigvee_{i\in I}\alpha_i:\bigvee_I S^n\to Y^{(n)}\]
with source and target $(n-1)$-connected CW-complexes. The map $\alpha$ induces an isomorphism on $H_n(-;\Z)$ because it sends the natural basis of $\vee_I S^n$ to the chosen basis of $H_n(Y^{(n)};\Z)$. Note also that source and target of $\alpha$ have trivial homology above dimension $n$. Hence $\alpha$ is a homology isomorphism between simply-connected CW-complexes, hence a homotopy equivalence.

By cellular approximation we can assume that $\alpha$ is cellular for the CW-structure on $\vee_I S^n$ with one $0$-cell and a $n$-cell for every $i\in I$. Then we form
\[Y'=Y\cup_{Y^{(n)}}\bigvee_I S^n\]
where the gluing is along a cellular homotopy inverse $g:Y^{(n)}\to \vee_I S^n$. The space $Y'$ then comes with a CW-structure with one $0$-cell and no cells in dimensions $1,\dots,\ni$ and the map
\[Y\to Y\cup_{Y^{(n)}}\bigvee_I S^n=Y'\]
is a homotopy equivalence (using the homotopy extension property).
\end{proof}
